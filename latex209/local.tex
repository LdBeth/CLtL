% local.tex -- released 26 February 1992
% Copyright(c) 1988,1992 by Leslie Lamport
%   for LaTeX version 2.09
%
% This file is used to produce a Local Guide for LaTeX users containing
% information specific to a site plus errors and omissions from the LaTeX
% manual (published by Addison-Wesley).
%
% This file may be changed and/or redistributed provided that it is
% stated clearly who the author is and to whom questions and complaints
% are to be sent.
%
% The installer of LaTeX at a site is responsible for customizing this
% document and providing copies for users.  He or she will have to read
% the text of this file CAREFULLY to see what must be added, removed, and
% changed.
 
% The \contact command is defined to generate the name of the person to
% whom questions should be sent.  This should be someone at the site.
% Most users' questions are easily answered by anyone slightly familiar
% with LaTeX or TeX. Don't bother anyone at another site with questions
% that can be answered locally.
 
\documentstyle[titlepage]{article}
 
\newcommand{\contact}{Leslie Lamport}
 
\newcommand{\BibTeX}{{\rm B\kern-.05em{\sc i\kern-.025em b}\kern-.08em
    T\kern-.1667em\lower.7ex\hbox{E}\kern-.125emX}}
 
 
\newcommand\bs{\char '134 }   % A backslash character for \tt font
\newcommand{\lb}{\char '173 } % A left brace character for \tt font
\newcommand{\rb}{\char '175 } % A right brace character for \tt font
 
\title{Using \LaTeX\ at SRC}
 
\author{Leslie Lamport}
 
\date{27 October 1988\\              % Keep this date current
For \LaTeX\ Version 2.09} 
 
\begin{document}
 
\maketitle
 
\tableofcontents
 
\newpage
 
\LaTeX\ runs on a variety of computers at many different sites.  This
document tells you how to use \LaTeX\ on the Magic, Lewis, and Clark
computers at SRC. It is not about \LaTeX\ itself, which is described by
the manual---{\em \LaTeX: A Document Preparation System}, published by
Addison-Wesley, available at fine book stores everywhere.
 
If you have a question that you can't answer by reading the manual and
this document, ask \contact.  He should also be informed of any
possible \LaTeX\ bugs or undocumented anomalies.
 
 
\section{Getting Started}
 
\subsection{Running a Sample File} \label{sec:sample}
 
Before preparing your own documents, you may want to get acquainted
with \LaTeX\ by running it on a sample input file.  First make your own
copy of the file \mbox{\tt sample.tex} by typing the following
Ultrix command:
\begin{verbatim}
     cp /usr/local/lib/tex82/sample.tex .
\end{verbatim}
(You must type the space followed by the period at the end.  This
and all other Ultrix commands are ended by typing {\em return}.)
A copy of the file \mbox{\tt sample.tex} is now in your current
directory; you can edit it just like any other file.  If you destroy or
mess up your copy, typing the above command again gets you a fresh
one.
 
Next, run \LaTeX\ on the file \mbox{\tt sample.tex} by typing:
\begin{verbatim}
     latex sample
\end{verbatim}
When \LaTeX\ has finished, it will have produced the file \mbox{\tt
sample.dvi} in your directory.  You can print this file by typing the
command
\begin{verbatim}
     iptex sample.dvi
\end{verbatim}
The output will be produced on the Imagen printer, located next to the
third floor receptionist's desk.  If you want to print \LaTeX\ output
on an Apple LaserWriter, see Section~\ref{sec:apple}.
 
After your output has been printed, you can delete \mbox{\tt
sample.dvi} by typing
\begin{verbatim}
     rm sample.dvi
\end{verbatim}
 
\subsection{Preparing and Running \LaTeX\ on Your Own Files}
 
You must use a text editor to prepare an input file for \LaTeX. The
document {\em Welcome to SRC\/} describes the text editors available
at SRC.  The easiest way to start learning about \LaTeX\ is by 
examining the file \mbox{\tt small.tex} with your text editor.
You can obtain your own copy of this file, in your directory,
by typing the command
\begin{verbatim}
     cp /usr/local/lib/tex82/small.tex .
\end{verbatim}
 
After you have prepared your file, whose name should have the extension
{\tt tex}, you must run it through \LaTeX\ and print the output.
Follow the instructions in Section~\ref{sec:sample}, except substitute
the first name of your file for ``\mbox{\tt sample}''.  Remember to
save disk space by deleting the {\tt dvi} file after printing the
output.
 
 
%List the text editors, available, and any special features they have
%for producing \LaTeX\ input.  Explain how the various text editors
%could cause bad characters to appear in the input file that would
%generate the
%\begin{verbatim}
%! Text line contains an invalid character.
%\end{verbatim}
%error.
 
If you want to stop \LaTeX\ in the middle of its execution, perhaps
because it is printing a seemingly unending string of uninformative
error messages, type {\em Control-C\/} (press $C$ while holding down
the key labeled {\em CTRL\/}).  This will make \LaTeX\ stop as if it
had encountered an ordinary error, and you can return to Ultrix command
level by typing {\tt X}, as described in the manual.  If typing {\em
Control-C\/} doesn't work, typing {\em Control-Z\/} will get you
immediately to Ultrix command level, but this will leave a stopped job
hanging around.  A stopped job won't hurt anything and will disappear
when you log out, but it forces you to type two successive \mbox{\tt
logout} commands to log out.
 
To use the {\em spell\/} program for finding spelling errors in a
\LaTeX\ input file named \mbox{\tt myfile.tex}, type the following
command:
\begin{verbatim}
     delatex myfile.tex | spell
\end{verbatim}
This will type a list of possibly misspelled words on your terminal.
If you'd rather have the output written to a file named \mbox{\tt
foo.bar}, type
\begin{verbatim}
     delatex myfile.tex | spell >foo.bar
\end{verbatim}
 
 
\section{Carrying On}
 
\subsection{\LaTeX\ on Ultrix} \label{sec:op-system}
 
The only special problems in using \LaTeX\ caused by the Ultrix
operating system involve the way Ultrix handles files.  The first
problem arises because, when a program starts to write a file, Ultrix
destroys the previous version of that file.  Thus, if an error forces
you to stop \LaTeX\ prematurely (by typing {\em Control-C\/} or {\em
Control-Z\/}), then the files that \LaTeX\ was writing are incomplete,
and the previous complete versions have been destroyed.  You probably
don't care about the output on the {\tt dvi} file, but, if you are
making a table of contents or using cross-referencing commands, then
\LaTeX\ also writes one or more {\em auxiliary files\/} that it reads
the next time it processes the same input file.  If the auxiliary files
are incomplete because \LaTeX\ was stopped before reaching the end of
its input file, then the table of contents and cross-references will be
incorrect the next time \LaTeX\ is run on the same input file.  You
will have to run \LaTeX\ a second time to get them right.  If you want
to avoid having to run \LaTeX\ twice after making an error---for
example, if your input is very long---then you should save copies of
these auxiliary files before running \LaTeX. An input file named
\mbox{\tt myfile.tex} and all the auxiliary files produced by \LaTeX\
from it are included in the Ultrix file specifier \mbox{\tt myfile.*}.
Use the Ultrix {\tt cp} command to save copies of these files.
 
The second problem in using \LaTeX\ on Ultrix involves the files that
\LaTeX\ reads.  The file whose name you type with Ultrix's {\tt latex}
command is called the {\em root file}.  In addition to reading the root
file, \LaTeX\ also reads the files specified by \hbox{\verb|\input|}
and \hbox{\verb|\include|} commands.  With the Ultrix directory system,
\LaTeX\ must know not only the names of these file but also on what
directories they are.  It will have no problem finding the correct
files if you follow two simple rules:
\begin{enumerate}
 \item Run \LaTeX\ from the directory containing the root file.
 \item Keep all files specified by \hbox{\verb|\input|} and 
      \hbox{\verb|\include|} commands in the same directory as the root
       file.
\end{enumerate}
If you follow these rules, you never have to type an Ultrix path
specifier when using \LaTeX.
 
You should never break the first rule, otherwise \LaTeX\ will have
trouble finding auxiliary files.  (To run \LaTeX\ on someone else's
file, copy the file to your directory.) If you break the second
rule, specifying a file from another directory in an
\hbox{\verb|\input|} or \hbox{\verb|\include|} command, you must use a
complete path name.  For example, to include the file \mbox{\tt hisfile.tex} 
from Jones' directory \hbox{\verb|/foo/bar|}, you can type
\begin{verbatim}
     \include{/udir/jones/foo/bar/hisfile}
\end{verbatim}
A \verb|~| character may not appear in the argument of an
\hbox{\verb|\input|} or \hbox{\verb|\include|} command, so you {\em
can't\/} use a file name such as \hbox{\verb|~jones/foo/bar/hisfile|}.
 
For people who don't like to obey rules, 
here is exactly how \LaTeX\ finds its
files.  The root file is found by Ultrix according to its usual rules.
\LaTeX's auxiliary files are read and written in the directory from
which it is run.  All file names specified in the \LaTeX\ input,
including the names of document-style ({\tt sty}) files specified by
the \hbox{\verb|\documentstyle|} command, are interpreted relative to
the directory from which \LaTeX\ is run.  If \LaTeX\ does not find a
file starting in this directory, it looks in the system directory
\hbox{\verb|/usr/local/lib/tex82|}.  You can change the directories in
which \LaTeX\ looks for its input files by setting the environment
variable \mbox{\tt TEXINPUTS}.  Putting the command
\begin{verbatim}
     setenv TEXINPUTS :.:/udir/jones/myown:/usr/local/lib/tex82:
\end{verbatim}
in your \mbox{\tt .login} file causes \LaTeX\ to look for files first
in the current directory, then in Jones' {\tt /myown} directory, and
then in the system directory.  You might want to do this if your name
is Jones and you have your own personal document-style files in your
{\tt /myown} directory. 
 
\subsection{Document Styles}
 
 
There are nine document styles and style options available at SRC that
are not described in the manual: 
\begin{itemize}
\item the \mbox{\tt proc} style option for making camera-ready copy for
conference proceedings,
 
\item The {\tt bezier} option for drawing curves.
 
\item The {\tt ifthen} option for implementing {\bf if-then-else} and
{\bf while-do} control structures.
 
\item The \mbox{\tt srcletter} style for making letters.
 
\item The {\tt showidx} option for printing index entries in the
margin.  
 
\item The {\tt makeidx} option for use with the {\it MakeIndex\/}
program. (Section~\ref{sec:makeindex} for information about this
program).
 
\item The {\tt ps} style option that uses Times Roman fonts.  
 
\item The {\tt preview} option for use with the {\em Proof\/}
 
\item The {{\tt ps-slides}} style for use with \SLiTeX\ to produce
color slides on the QMS color printer.
\end{itemize} 
The {\tt ps} option is described in Section~\ref{sec:apple}; the
remaining stye options are described below.
 
 
\subsubsection{The {\tt proc} Style Option}
 
The {\tt proc} option is used with the \mbox{\tt article} document
style.  It produces two-column output for ACM and IEEE conference
proceedings.  The command \hbox{\verb|\copyrightspace|} makes the blank
space at the bottom of the first column of the first page, where the
proceedings editor will insert a copyright notice.  This command works
by producing a blank footnote, so it is placed in the text of the first
column.  It must go after any \hbox{\verb|\footnote|} command that
generates a footnote in that column.
 
\LaTeX\ automatically numbers the output pages.  It's a good idea 
to identify the paper on each page of output.  Placing the command
\begin{verbatim}
     \markright{Jones---Foo}
\end{verbatim}
in the preamble (before the \hbox{\verb|\begin{document}|} command)
prints ``Jones---Foo'' at the bottom of each page.  
 
\subsubsection{The {\tt bezier} Style Option}
 
This option defines a single command, \hbox{\verb|\bezier|}, that draws
a curved line in a {\tt picture} environment.  Let $P_{i}$ be the point
with coordinates $(x_{i},y_{i})$, for $i=1$, 2, and 3.  The command
\begin{itemize} \tt
\item[]
\verb|\bezier{|$n$\verb|}(|$x_{1}$,$y_{1}$)($x_{2}$,$y_{2}$)($x_{3}$,$y_{3}$)
\end{itemize}
draws $n$ points on the quadratic Bezier spline determined by the three
points $P_{1}$, $P_{2}$, and $P_{3}$.  The locus of points on this
spline is a parabolic arc from $P_{1}$ to $P_{3}$ having the line
$P_{1}P_{2}$ tangent to it at $P_{1}$ and the line $P_{2}P_{3}$ tangent
to it at $P_{3}$.  Note that $P_{2}$ is {\em not\/} on this arc unless
$P_{1}$, $P_{2}$, and $P_{3}$ are colinear, in which case the arc is a
straight line.  Bezier splines are useful because it's easy to join two
of them together smoothly by giving them the same tangent line where
they meet.
 
It takes roughly 75 points per inch to form a solid line, depending
upon the line thickness.  See Section~C.13.3 of the manual for commands
to specify line thickness in a {\tt picture} environment.  This command
is {\em very\/} slow, and \TeX\ has enough memory to hold only about
1000 points plus a page of text.  (Remember that \TeX\ keeps the
current page plus all as yet unprinted figures in memory.) So, the
\verb|bezier| command should be used for only a small number of small
curves.
 
 
\subsubsection{The {\tt ifthen} Style Option}
 
This option provides two programming language features that are useful
only for people who already know how to program.  It defines the
two commands
\begin{itemize} 
\item[]
\verb|\ifthenelse{|{\em test\/}\verb|}{|{\em then clause\/}\verb|}{|%
{\em else clause\/}\verb|}|\\
\verb|\whiledo{|{\em test\/}\verb|}{|{\em do clause\/}\verb|}|
\end{itemize}
that implement the following two Pascal language structures
\begin{itemize}
\item[]
\begin{tabbing}
{\bf if} {\em test\/} \= {\bf then} \= {\em then clause\/} \\
 \> {\bf else} \> {\em else clause\/} \\[2pt]
{\bf while} {\em test\/} {\bf do} {\em do clause\/}
\end{tabbing}
\end{itemize}
The {\em then\/}, {\em else\/}, and {\em do\/} clauses
are ordinary \LaTeX\ input; {\em test\/} is one of the following:
\begin{itemize}
 \item A relation between two numbers formed with {\tt <}, {\tt >},
or {\tt =}; for example, \hbox{\verb|\value{page}>3|}.
 \item \verb|\equal{|{\em string1\/}\verb|}{|%
{\em string2\/}\verb|}|, which evaluates to {\em true\/} if {\em
string1\/} and {\em string2\/} are the same strings of characters after
all commands have been replaced by their definitions.  (Upper- and
lowercase letters are unequal.)
 \item A logical combination of the above two kinds of tests
    using the operators \hbox{\verb|\or|}, \hbox{\verb|\and|},
    and \hbox{\verb|\not|} and the parentheses \hbox{\verb|\(|}
    and \hbox{\verb|\)|}---for example:
\begin{verbatim}
\not \( \value{section} = 1  \and  \equal{Jones}{\myname} \)
\end{verbatim}
\end{itemize}
The {\em test\/} argument is a violently moving argument, which means
that not only fragile commands but even some commands that are not
normally fragile will break, causing \TeX\ to enter an infinite loop.
The \verb|\protect| command works in these situations.
 
These commands, together with \hbox{\verb|\renewcommand|} and the
commands of Section~C.7.4 for manipulating counters, open up a whole
new world of hacking.
 
 
\subsubsection{Letters} \label{sec:letters}
 
The \mbox{\tt letter} document style, described in the manual, should
be used for generating personal letters.  For generating letters to be
copied onto SRC letterhead, use the \mbox{\tt srcletter} style.  This
style works for the old letterhead (the one with the Digital logo
printed in blue).  Mailing labels are formatted in two columns of five
$2''\times\mbox{4-1/4}''$ labels each, suitable for copying onto Avery
brand, number 5352 address labels.
 
There are no features for making letters other than those described in
the manual.  However, suggestions will be accepted for such options as
the ability to print ones net address as part of the letterhead.
 
\subsubsection{The {\tt showidx} Style Option}
 
This style option, for use with the {\tt report} or {\tt book} document
styles, causes index entries to be printed in the outer margin.  It
does not change the effect of \verb|\makeindex|, which controls
whether or not an {\tt .idx} file is written.  No attempt is made to
avoid overprinting marginal notes.  This option issues a 
\verb|\flushbottom| declaration.
 
 
\subsubsection{The {\tt preview} Style Option}
 
This option formats the output so it will appear as a continuous scroll
when viewed with the {\em Proof\/} preview program.  (See the {\em man\/}
page for for a description of {\em Proof}.)
 
\subsection{The {\tt ps-slides} Style} \label{sec:ps-slides}
 
This style causes \SLiTeX\ to produce output that can be printed on the
QMS color printer with {\tt aptex} to produce color transparencies.
When the output is printed on a black-and-white PostScript printer,
colors are printed as different shades of gray.
 
The {\tt ps-slides} style has the following differences from the
ordinary {\tt slides} style.
\begin{itemize}
\item The \verb+\colors+
 command is made a no-op. Colors that are not predefined
    must be defined with the \verb+\newpscolor+ command (see below).
 
\item The \verb+\colorslides+ and \verb+\blackandwhite+ commands are
the same, except that
\verb+\colorslides+ prints only slides and overlays, not notes.
 
\item The meaning of the \verb+\invisible+ declaration has been
changed, as described below.  The old \verb+\visible+ command is the
same as the new \verb+\white+ command.
 
\item The default page style is {\tt plain}.  (The alignment marks are
no longer of any use.)
 
\item The positioning of the body of the slide and the slide number have
   been changed to accomodate the QMS printer's small effective page 
   size.
\end{itemize}
Except for these differences, a \SLiTeX\ input file prepared for the
ordinary {\tt slides} style {\em should\/} work with the {\tt
ps-slides} style.  There are probably some \SLiTeX\ commands that will
interact incorrectly with color-changing commands.  These problems will
be corrected when reported.  The most likely bug is a color declaration
``escaping from its scope'', having a more global effect than it should.
Such a problem can be worked around by adding redundant color-changing
commands.
 
The new commands defined by the {\tt ps-slides} style are listed below.
\begin{description}
\item[{\tt 
  \begin{tabular}[b]{@{}l@{}}
      \bs newpscolor\lb{\em cmd\/}\rb
          \lb{\em red\/}\rb\lb{\em green\/}\rb\lb{\em blue\/}\rb \\
      \bs renewpscolor\lb{\em cmd\/}\rb\lb{\em red\/}\rb
          \lb{\em green\/}\rb\lb{\em blue\/}\rb
  \end{tabular}}] \mbox{}\\*
Define the command {\em cmd\/} to produce a new color with the
indicated RGB values, where {\em red\/}, {\em green\/}, and {\em
blue\/} are decimal numbers between 0 and 1 (inclusive).  For example:
\begin{quote}
\begin{tabular}{@{}l@{ $=$ [{\em red\/}: }l@{, {\em green\/}: }l@{, {\em 
  blue\/}:}l@{]}}
           Black         &  0  & 0  & 0\\
           White         &  1  & 1  & 1\\
           Red           &  1  & 0  & 0\\
           Light yellow  &  .5 & .5 & 0
\end{tabular}
\end{quote}
For \verb+\newpscolor+, {\em cmd\/} must not already be defined; for
\verb+\renewpscolor+, it must already be defined (but not necessarily
as a color).  
 
The following colors are predefined: 
\begin{quote}
     \verb+\black+, \verb+\red+, \verb+\green+, \verb+\blue+,
      \verb+\yellow+,  \verb+\magenta+, \verb+\cyan+, 
      \verb+\white+~\hspace*{-30pt}\mbox{}
\end{quote}
For ordinary printers, which assume white paper, \verb+\white+ text is
invisible.  Unlike in ordinary \SLiTeX, color-changing commands can be
used in math mode.
 
\item[{\tt
\begin{tabular}[b]{@{}l@{}}
 \bs invisible\\ 
 \bs visible   
\end{tabular}}] \mbox{}\\*
Ordinary declarations of visibility of the text.  Invisible text
is not printed.  (\TeX\ prints it in a font that whose letters are all
print as spaces.)
Visibility is independent of color, so you can have invisible red text,
visible green text, and even visible white text.  (Visible white text
can be seen only when printed by a weird printer that prints on
nonwhite paper.)
 
\item[{\tt
\begin{tabular}[b]{@{}l@{}}
 \bs norestore\\ 
 \bs restore
\end{tabular}}] \mbox{}\\*
The \verb+\norestore+ declaration inhibits the proper scoping of color
declarations, causing color declarations to act as if they were global.
The \verb+\restore+ declaration causes the current and future colors to
become the ones they would have been had there been no \verb+\norestore+
command.  Both \verb+\norestore+ and \verb+\restore+ are global declarations.
Example:
\begin{verbatim}
    \black black {\green green 
                         \norestore {\red red} red
                         \restore 
                         green} 
           black
\end{verbatim}     
The \verb+\norestore+ declaration is handy for color commands inside a
tabbing or tabular environment.  However, there are some anomalies:
 
\begin{itemize}
\item A \verb+\restore+ command inside a tabbing environment may not
work exactly as it should.  Try putting the command immediately after a
\verb+\=+, \verb+\\+, or \verb+\>+. 
 
\item An \verb+\fbox+, \verb+\framebox+, or \verb+\frame+ command may
do weird things in the scope of a \verb+\norestore+ if there's a color
declaration in its argument.
     
\item A \verb+\background + may behave strangely inside the scope of a
 \linebreak %%%%%%
\verb+\norestore+.
\end{itemize}     
 
\item[{\tt \bs background\lb{\em color\/}\rb\lb{\em text\/}\rb }]
\mbox{}\\*
Typesets {\em text} in an \mbox with a box behind it whose color is
determined by the {\em color\/} color-changing command.  The yellow box
is the width of {\em text\/} plus a border of width \verb+\bgborder+
around it.  
 
\item[{\tt
\begin{tabular}[b]{@{}l@{}}
 \bs hollowbackground \\
 \bs filledbackground
\end{tabular}}] \mbox{}\\*
A \verb+\hollowbackground+ declaration causes any 
\verb+\background+
command in its scope to produce an outline, much like \verb+\fbox+,
instead of a filled box.  This is useful for checking the slides with
Proof or on a black-and-white printer.  A \verb+\filledbackground+
declaration has the opposite effect.  
 
\item[{\tt \bs nogray}] \mbox{}\\*
A declaration that makes every color other than
\verb+\white+ be equivalent to \verb+\black+, and issues a
\verb+\hollowbackground+ declaration.  Used to print the slides on a
black-and-white printer so colors come out black instead of various
shades of gray.
\end{description}
 
\subsection{Where the Files Are}
 
% must explain where the following files are:
%   small.tex, sample.tex, *.sty, *.doc, lablst.tex, idx.tex
%
 
All \LaTeX\ files mentioned in the manual, including the {\tt sty} and
{\tt doc} files, are in the directory \mbox{\tt /usr/local/lib/tex82}.
Fonts are stored in two directories: the {\tt tfm} files used by \TeX\
are on \mbox{\tt /usr/local/fonts/tfm}, and the pixel files used by
{\em iptex\/} and other device drivers are in \mbox{\tt
/usr/local/fonts/pixel}.
 
\subsection{Running {\tt lablst.tex} and {\tt idx.tex}}
 
A list of labels and citations in an input file is printed
by running \LaTeX\ on the input file \mbox{\tt lablst.tex},
which is done by typing
\begin{verbatim}
     latex /usr/local/lib/tex82/lablst
\end{verbatim}
\LaTeX\ will then ask for the name of the input file, which should be
typed without an extension, and for the name of the main document style
(e.g., \mbox{\tt article}), used by that file.
 
The index entries on an {\tt idx} file are printed by running \LaTeX\
on the file \mbox{\tt idx.tex}, which is done by typing
\begin{verbatim}
     latex /usr/local/lib/tex82/idx
\end{verbatim}
\LaTeX\ will ask for the name of the {\tt idx} file, which is typed
without an extension.
 
 
 
\subsection{Differences from the Manual}
 
All \LaTeX\ features described in the manual are provided by 
the implementation at SRC.
 
%Explain here any characters that can appear in input files other than
%the ones listed in Section 2.1.
 
%Tell if the \mbox{\tt log} file has an extension other than
%\mbox{\tt .log}.  Note: on TOPS-20, its extension is \mbox{\tt .lst}.
 
%Describe the sizes of disks and circles the are available.
 
%Don't forget to mention if the invisible fonts needed for \SLiTeX\
%color slides are unavailable.
 
\subsection{Using \BibTeX}
 
\BibTeX\ is a program for compiling a reference list for a document
from a bibliographic database.  It is run by typing
\begin{verbatim}
     bibtex myfile
\end{verbatim}
where \mbox{\tt myfile.tex} is the name of your \LaTeX\ input file.
This reads the file \mbox{\tt myfile.aux}, which was generated when you
ran \LaTeX\ on \mbox{\tt myfile.tex}, and produces the file \mbox{\tt
myfile.bbl}.  \BibTeX\ should be run from the directory containing
\mbox{\tt myfile.tex} (which should be the same directory from which
\LaTeX\ was run on that file).
 
If the {\tt bib} file is not in the same directory as the \LaTeX\ input
file---for example, if you're using someone else's {\tt bib}
file---then you must include a path as part of the file name specified
by the \hbox{\verb|\bibliography|} command.  A \verb|~| cannot appear
in the argument of a \hbox{\verb|\bibliography|} command, so you should
use a complete path name.  For example, the \LaTeX\ command
\begin{verbatim}
     \bibliography{/udir/jones/bibfiles/gnus}
\end{verbatim}
specifies the file \mbox{\tt gnus.bib} kept by Jones in his 
\mbox{\tt /bibfiles} directory.
 
 
There is now no formal provision for sharing bibliographic database
information, nor are there programs to assist in making your own {\tt
bib} files.  Suggestions for forming one or more common {\tt bib} files
are welcome.
 
In addition to the bibliography styles described in the manual, there
is a {\tt ieeetr} style that formats entries in the style of the IEEE
transactions.
 
In addition to the usual three-letter abbreviations for the months, the
following abbreviations are defined by the bibliography styles:
\begin{list}{}{\labelwidth 0pt \itemindent-.5\leftmargin
       \itemsep=2pt plus 1pt
       \let\makelabel\descriptionlabel}\it
\item[\tt acmcs] ACM Computing Surveys
\item[\tt acta] Acta Informatica
\item[\tt cacm] Communications of the ACM
\item[\tt ibmjrd] IBM Journal of Research and Development
\item[\tt ibmsj] IBM Systems Journal
\item[\tt ieeese] IEEE Transactions on Software Engineering
\item[\tt ieeetc] IEEE Transactions on Computers
\item[\tt ieeetcad]
 IEEE Transactions on Computer-Aided Design of Integrated Circuits
\item[\tt ipl] Information Processing Letters
\item[\tt jacm] Journal of the ACM
\item[\tt jcss] Journal of Computer and System Sciences
\item[\tt scp] Science of Computer Programming
\item[\tt sicomp] SIAM Journal on Computing
\item[\tt tocs] ACM Transactions on Computer Systems
\item[\tt tods] ACM Transactions on Database Systems
\item[\tt tog] ACM Transactions on Graphics
\item[\tt toms] ACM Transactions on Mathematical Software
\item[\tt toois] ACM Transactions on Office Information Systems
\item[\tt toplas] ACM Transactions on Programming Languages and Systems
\item[\tt tcs] Theoretical Computer Science
\end{list}
 
% Note: All styles should share the same set of abbreviations.
 
\subsection{Using \SLiTeX}
 
\SLiTeX\ is a version of \LaTeX\ for making slides.
To run \SLiTeX\  with a root file \mbox{\tt myroot.tex}, you type
\begin{verbatim}
     slitex myroot
\end{verbatim}
(You should be connected to the directory containing \mbox{\tt
myroot.tex}.) Refer to Section~\ref{sec:op-system} if you want slide
files or \hbox{\verb|\input|} files to be in a different directory from
your root file.
 
Color slides can be made by copying the color layers produced by \SLiTeX\
onto color transparencies using the Thermofax machine on the second
floor.  However, it is easier to make them with the QMS color printer 
on the 3rd floor.  To use it, you must use the {\tt ps-slides} 
document style described in Section~\ref{sec:ps-slides}.  The output
of \SLiTeX\ is then printed by the command
\begin{quote} \tt
     aptex -Pcolor {\em file-name}.dvi
\end{quote}
Before executing this command, check that the printer has
transparencies in the paper tray and perform the appropriate protocol
to synchronize with other users of the printer.
 
\subsection{Using {\em MakeIndex\/}} \label{sec:makeindex}
 
The {\em MakeIndex\/} program helps in making an index.  It is 
described in a separate document, available through the
{\tt printdoc} command.
 
 
\subsection{Fonts}
 
Almost all the symbols available on our fonts can be generated by
ordinary \LaTeX\ commands.  However, there are type sizes not
obtainable by \LaTeX's size-changing commands with the ordinary
document styles.  Consult a local \TeX\ expert to find the
\TeX\ name for such a font.
 
Tables~\ref{tab:styles} and \ref{tab:fonts} allow you
to determine if the font for a type style at a particular
size is preloaded, loaded on demand, or unavailable.
\begin{table}
\centering
\begin{tabular}{l|r|r|r|}
\multicolumn{1}{l}{size} & 
\multicolumn{1}{c}{default (10pt)} &
        \multicolumn{1}{c}{11pt option}  &
        \multicolumn{1}{c}{12pt option}\\
\cline{2-4}
\verb|\tiny|       & 5pt  & 6pt & 6pt\\
\cline{2-4}
\verb|\scriptsize| & 7pt  & 8pt & 8pt\\
\cline{2-4}
\verb|\footnotesize| & 8pt & 9pt & 10pt \\
\cline{2-4}
\verb|\small|        & 9pt & 10pt & 11pt \\
\cline{2-4}
\verb|\normalsize| & 10pt & 11pt & 12pt \\
\cline{2-4}
\verb|\large|      & 12pt & 12pt & 14pt \\
\cline{2-4}
\verb|\Large|      & 14pt & 14pt & 17pt \\
\cline{2-4}
\verb|\LARGE|      & 17pt & 17pt & 20pt\\
\cline{2-4}
\verb|\huge|       & 20pt & 20pt & 25pt\\
\cline{2-4}
\verb|\Huge|       & 25pt & 25pt & 25pt\\
\cline{2-4}
\end{tabular}
\caption{Type sizes for \LaTeX\ size-changing commands.}\label{tab:styles}
\end{table}
\begin{table}
\centering
\begin{tabular}{l|c|c|c|c|c|c|}
\multicolumn{1}{l}{}& 
\multicolumn{1}{c}{\tt \bs it} &
\multicolumn{1}{c}{\tt \bs bf} &
\multicolumn{1}{c}{\tt \bs sl} &
\multicolumn{1}{c}{\tt \bs sf} &
\multicolumn{1}{c}{\tt \bs sc} &
\multicolumn{1}{c}{\tt \bs tt} \\
\cline{2-7}
5pt  & D & D & X & X & X & X \\
\cline{2-7}
6pt  & X & D & X & X & X & X \\
\cline{2-7}
7pt  & P & D & X & X & X & X \\
\cline{2-7}
8pt  & P & D & D & D & D & D \\
\cline{2-7}
9pt  & P & P & D & D & D & P \\
\cline{2-7}
10pt & P & P & P & P & D & P \\
\cline{2-7}
11pt & P & P & P & P & D & P \\
\cline{2-7}
12pt & P & P & P & P & D & P \\
\cline{2-7}
14pt & D & P & D & D & D & D \\
\cline{2-7}
17pt & D & P & D & D & D & D \\
\cline{2-7}
20pt & D & D & D & D & D & D \\
\cline{2-7}
25pt & X & D & X & X & X & X \\
\cline{2-7}
\end{tabular}
\caption{Font classes: P = preloaded, D = loaded on demand, 
         X = unavailable.}\label{tab:fonts}
\end{table}
Table~\ref{tab:styles} tells you what size of type is used for each
\LaTeX\ type-size command in the various document-style options.  For
example, with the {\tt 12pt} option, the \hbox{\verb|\large|}
declaration causes \LaTeX\ to use 14pt type.  Table~\ref{tab:fonts}
tells, for every type size, to which class of fonts each type style
belongs.  For example, in 14pt type, \verb|\bf| uses a preloaded
font and the other five type-style commands use load-on-demand fonts.
Roman (\verb|\rm|) and math italic (\verb|\mit|) fonts are all
preloaded; the \hbox{\verb|\em|} declaration uses either italic
(\verb|\it|) or roman.
 
%Describe any special fonts available here.
 
%Tell where to find a font manual, listing the character numbers of
%the symbols on different fonts.
 
%Include a table telling which fonts are preloaded, loaded on demand,
%and unavailable.  
 
\subsection{Using Times Roman Fonts} \label{sec:apple}
 
The fonts normally used by \LaTeX\ are from the Computer Modern Roman
font family designed by Donald Knuth.  Some people prefer the Times
Roman family of fonts, which are available on the Apple LaserWriter
printers.  They are obtained with the {\tt ps} document-style option.
Currently, this option does not work with the {\tt 12pt} style option.
Also, fonts from the Times Roman family are available only for the
default \verb|\rm| style and for the \verb|\it|, \verb|\bf|, and
\verb|\sf| styles.  Other styles use the ordinary CMR fonts.
 
\TeX\ can use only the CMR fonts in math mode, so \verb|${\rm foo}=3$|
produces ``foo'' in CMR font, which is probably not what you want.
Instead, use \verb|$\mbox{\rm foo}=3$|, which typesets ``foo'' in Times
Roman because \TeX\ is not in math mode inside the \verb|\mbox|.  No
one will notice that ordinary math symbols like $\alpha$ or $x$ are
typeset in a CMR font rather than a Times Roman font.
 
A {\tt dvi} file produced with the {\tt ps} option can be printed only
on the Apple LaserWriter using the {\tt aptex\/} program.  Type the
Ultrix command {\tt man~aptex} to find out how to run this program.
You can print any {\tt dvi} file with {\tt aptex}, but it is slower
than {\tt iptex} and doesn't do a very good job printing the
standard CMR fonts.  So, use {\tt aptex} to print only {dvi} files
created with the {\tt ps} option.
 
The {\tt ps} option causes \LaTeX\ to use more fonts than usual,
so there's a good chance that you'll run \TeX\ out of font
space if you use lots of different fonts.
  
\subsection{Special Versions}
 
No foreign-language or other special versions of \LaTeX\
are currently available at SRC.
 
 
\section{Bugs}
 
There are a few known bugs in \LaTeX\ that occur very seldom and
cause the user little trouble, but would be very difficult to fix.
Moreover, given the nature of complex systems, it is not unlikely that
the corrections would lead to even worse problems.  Therefore, these
bugs will probably not be fixed.  
 
The bugs and ways to get around them are listed below.  Do not worry
about any of them until you are preparing the final draft, since
changes to the text are very likely to cause the problem to disappear.
\begin{itemize}
\item In rare instances, a figure or table will be printed on the page
preceding the text where the {\tt figure} or {\tt table} environment
appears.  This can be fixed by either moving the environment further
towards the end of the document. 
 
\item A marginal note at the top of a page may appear in the wrong
margin.  This can be fixed by inserting a redundant \verb|\pagebreak|
command to force a page break exactly where \LaTeX\ started the new
page anyway.
 
\item A footnote can be broken across two pages when it should fit on a
single page.  This happens when there is one or more figures or tables
on the page.  The problem is corrected by moving, towards the end of the
file, the last {\tt figure} or {\tt table} environment that produces a
figure or table on the page where the footnote starts.
\end{itemize}
 
 
\section{Errata and Additions to the Manual}
 
% Errata and additions to the first edition of the LaTeX manual,
% published by Addison-Wesley.
% 31 October 1990
 
These are all the errors and omissions to the manual, \LaTeX: {\em A
Document Preparation System} reported by 31 October 1990.
 
\subsection{First Four Printings}
 
The following errors were corrected in the Fifth Printing.  To see if
your copy is one of the first four printings, turn to the copyright
page---the one with the Library of Congress cataloging information, on
the back of the title page.  If the fifth line on that page begins
``that, as of August 27, 1985'', then your copy is from the first four
printings.
 
\subsubsection*{page xiii}
 
Add Mike Urban to the list of people thanked in the third paragraph.
% Mike Urban
 
\subsubsection*{page 2}
 
In the first paragraph of Section~1.1, replace the three instances of
{\tt sample.tex} by {\tt small.tex}.
% LL, 8 Jan 86
 
\subsubsection*{page 15, line 18}
 
Replace ``thay'' by ``that''.
% Mike Urban, 4 Jan 86
 
\subsubsection*{page 24, line 13}
Add a period to the end of the line.
% Edgar Whipple, 10 Apr 86
 
\subsubsection*{page 31, line 7}
Delete one ``will''.
% Nelson Beebe, 27 Feb 86
 
\subsubsection*{page 39, line 2}
 
Change ``six'' to ``five''.
% Stephen Gildea, 14 Feb 86
 
\subsubsection*{page 39, line 17}
Remove the extra ``it''.
 
\subsubsection*{page 40, line $-8$}
Change ``five'' to ``six''.
% Hal Perkins 25 Mar 86
 
\subsubsection*{Section~3.3.5, page 49}
The name of the environment is {\tt eqnarray}.  There are two instances
on this page of the incorrect name ``{\tt eqnarry}'' that should be
changed.
% Oren Patashnik
 
 
\subsubsection*{page 52, last line}
 
Replace ``instead of $\int\int ydxdy$'' by 
``instead of $\int\int zdxdy$''.
% Oren Patashnik 16 Jan 86
 
\subsubsection*{pages 55--58, 124, and 173--174}
 
Commands that define or redefine a command or environment, such as the
\linebreak
\hbox{\verb|\newcommand|} and \hbox{\verb|\renewenvironment|} commands,
should not be nested within one another.  Doing so may result in the
following \TeX\ error:
\begin{verbatim}
! Illegal parameter number in definition of ... .
\end{verbatim}
 
\subsubsection*{page 56, line $-3$}
 
Change ``\verb|{\bf x}|'' to ``\verb|{x}|''.
% Stephen Gildea, 14 Feb 86
 
\subsubsection*{page 60, line $-7$}
 
Replace ``one of these environments'' by ``a {\tt figure} or
{\tt table} environment''.
% improvement suggested by Arthur Keller, 1 Jan 86
 
\subsubsection*{page 74, line 20}
Change \verb|\begin{documentstyle}| to \verb|\begin{document}|.
% Hal Perkins 25 Mar 86
 
 
\subsubsection*{page 75, line 15}
Change \verb|\thebibliography|  to \verb|\bibliography|.
% Charles Karney 6 Jan 86
 
\subsubsection*{page 79, line 16}
Change ``diplaying'' to ``displaying''.
% Stephen Gildea, 14 Feb 86
 
\subsubsection*{page 88, line 27}
 
Change ``entire the paragraph''to ``the entire paragraph''.
 
\subsubsection*{page 89, line 16}
 
Change ``page-breaking'' to ``line-breaking''.  
% Arthur Keller, 1 Jan 86
 
\subsubsection*{page 95, line 5}
 
Replace ``\verb|{.01in}|'' by ``\verb|{1.01in}|''.
 
\subsubsection*{page 96}
 
In the penultimate paragraph of Section~5.4.2, replace 
\begin{quote}
The \verb|\vfill| command is an abbreviation for \verb|\vspace{\fill}|.
\end{quote}
with the following:
\begin{quote}
The \verb|\vfill| command is equivalent to a blank line followed by
\verb|\vspace{\fill}|.
\end{quote}
 
\subsubsection*{page 98, line 6}
 
Remove an ``i'' from ``directiion''. 
% Arthur Keller, 1 Jan 86
 
\subsubsection*{page 102}
In Figure~5.1, replace the two occurrences of ``-1.8'' by
``$-1.8$''.
% Charles Karney, 6 Jan 86       
 
\subsubsection*{page 104, line $-3$}
Correct the spelling of ``positioning''.
% Nelson Beebe, 27 Feb 86
 
% \subsubsection*{page 108}
% There is an insignificant error in the picture of in the first
% example box.
 
\subsubsection*{page 113}
{\sloppy Change the label on the arrow above Item~2 from \verb|\itemsep| to
\verb|\itemsep + \parsep|.\par}
% Stephen Gildea 24 Mar 86
 
 
\subsubsection*{page 116, line 20}
 
Change ``fourteen-point Plus Roman'' to ``twelve-point Plus Roman''.
% Arthur Keller, 1 Jan 86
 
\subsubsection*{page 118, line $-10$}
 
Change this line to:
\begin{small}
\begin{verbatim}
... (myfile.tex [1] [2] [3] (part1.tex [4] [5]) (part2.tex [6] [7]
\end{verbatim}
\end{small}
% Arthur Keller, 1 Jan 86
 
\subsubsection*{page 129}
 
The following \LaTeX\ warning messages should be listed:
\begin{itemize}
\item[] {\tt No ... typeface in this size, using ...}\\
A type style declaration specified a type style and size combination
that is not available, so \LaTeX\ is substituting another one.
 
\item[] {\tt ... in math mode.}\\
The indicated command is not permitted in math mode but was used there.
Remember that \verb|\boldmath|, \verb|\unboldmath|, and size-changing
commands may not be used in math mode.
\end{itemize}
% Stephen Gildea, 14 Feb 86
 
 
\subsubsection*{pages 141--142}
 
Change ``von Beethoven'' to ``van Beethoven''.
% Louis E. Janus 15 Dec 86
 
 
\subsubsection*{page 142}
 
Add the following near the bottom of the page, just above the
{\bf Titles} heading.
\begin{quote}
If an {\tt author} or {\tt editor} field has more names than you want to
type, just end the list of names with \index{and others}{\tt and
others}; the standard styles convert this to the conventional {\em et
al.}
\end{quote}
 
\subsubsection*{page 147, line 6}
 
Change ``Scribe-compatible'' to ``{\em Scribe\/}-compatible''.
% Oren Patashnik 10 Feb 86
 
\subsubsection*{page 152, line $-6$}
Add \verb|\newtheorem| and \verb|\hyphenation| to the list
of global declarations.
% Hal Perkins 25 Mar 86
 
 
\subsubsection*{page 157, line $-10$}
The first paragraph of Section~C.3.1 should note that the {\tt *}-form
of a sectioning command does not affect the running head.
% Oren Patashnik 24 Mar 86
 
 
\subsubsection*{page 158, line $-12$}
Change ``chapters'' to ``appendix chapters''.
% Hal Perkins 25 Mar 86
 
\subsubsection*{page 160, top line}
 
Change ``(Section 5.6)'' to ``(Section 5.3)''.
 
\subsubsection*{page 160, line $-9$}
 
Change the description of the {\tt openbib} style option to: ``Causes
the bibliography (Section 4.3) to be formatted in open style.  (See
van~Leunen~[7].)''
% Charles Karney, 6 Jan 85
 
 
\subsubsection*{page 163, line 19}
Change ``Section 5.1.2'' to ``Section 5.1.1''.
% Hal Perkins 25 Mar 86
 
\subsubsection*{page 168, line $-5$}
 
Replace ``printed as \verb*|\ |'' by ``printed as \verb*| |''.
 
\subsubsection*{page 169, line $-5$}
Change ``first and third rows'' to ``first and third columns''.
 
\subsubsection*{page 169, line $-3$}
Add a period to the end of the sentence.
 
\subsubsection*{page 170}
 
Add the following paragraph after line 5:
\begin{quote}
An overfull \verb|\hbox| warning occurs if a formula extends beyond the
prevailing margins.  However, if the formula does lie within the
margins, no warning is generated even if it extends far enough to
overprint the equation number.
\end{quote}
 
\subsubsection*{page 174, line $-1$}
Add the following sentence: ``It is a global declaration.''
% Hal Perkins 25 Mar 86
 
\subsubsection*{page 176}
 
Change the four lines immediately following the heading for
Section C.8.1 to:
\begin{quote}
\begin{tabular}{lll}
 \verb|\begin{figure}[|{\em loc\/}{\tt ]} & {\em body\/}
      & \verb|\end{figure}| \\
 \verb|\begin{figure*}[|{\em loc\/}{\tt ]} & {\em body\/}
      & \verb|\end{figure*}| \\
 \verb|\begin{table}[|{\em loc\/}{\tt ]} & {\em body\/}
      & \verb|\end{table}| \\
 \verb|\begin{table*}[|{\em loc\/}{\tt ]} & {\em body\/}
      & \verb|\end{table*}| 
\end{tabular}
\end{quote}
 
% \subsubsection*{page 182, Figure C.5}
% Interchange ``high'' and ``low''
% Stephen Gildea, 14 Feb 86
 
\subsubsection*{page 185, line $-5$}
Replace ``suppresses command'' by ``command suppresses''.
%Rusty Wright 5 Dec 85
 
\subsubsection*{page 187, lines 15--17}
The sentence ``It also writes {\em bib\_files\/}\ldots'' is
redundant and can be eliminated.
% Charles Karney, 6 Jan 86
 
\subsubsection*{page 188, lines 1, 2, 4, 8} 
Change ``{\em key\_lst\/}'' to ``{\em key\_list\/}''.
% Dave Johnson, 10 Mar 86
 
\subsubsection*{page 191, line 13} 
The word ``paragraph'' misspelled.
% Mark Kent 9 Dec 85
 
\subsubsection*{page 191, line 14}
Remove the space between \verb|\begin| and \verb|{sloppypar}|.
(This is for consistency only; \LaTeX\ ignores the space.)
 
\subsubsection*{page 199}
Add the following sentence to the last paragraph on the page:
\begin{quote}
Words typeset in typewriter style or in two different styles
are not hyphenated except where permitted by \verb|\-| commands.
\end{quote}
(This is a change to \LaTeX\ made on 18~December~1985.) Also, add the
following index entry citations to this page: ``\verb|\-|'',
``hyphenation, suppressed'', and ``typewriter type style, no
hyphenation in''.
 
\subsubsection*{page 217}
Add the subentry ``{\tt openbib}, 160'' to the index entry
``document-style option''.
 
 
\subsubsection*{page 223, column 2, line 13}
Change ``mad'' to ``made''.
% Barbara Beeton, 23 Apr 86
 
\subsubsection*{page 225}
Add page 47 to the index entry for ``Leunen, Mary-Claire van''.
% Oren Patashnik
 
\subsubsection*{page 228}
In index entry for \verb|\multicolumn|, change ``194''
to ``184''.
 
 
\subsubsection*{Tear-Out Command Sheet}
 
In the first column, sixth line after ``Sentences and Paragraphs'' heading, 
replace ``\hbox{\& \verb|&|}'' by ``\hbox{\& \verb|\&|}''.
 
\subsection{First Five Printings}
 
\subsubsection*{page xiii, line $-7$}
The hyphenation of ``Leunen'' is obviously incorrect.
% Don Knuth 11 May 89
 
\subsubsection*{page7, line 33}
Add the following sentence after ``[6].''
\begin{quote}
(Many of the details of \LaTeX, including command names, were also
taken from {\em Scribe}.)
\end{quote}
 
\subsubsection*{page 12, line $-7$}
Replace ``every'' with ``ever''.
% David Sullivan, 13 Apr 88
 
\subsubsection*{page 14}
{\sloppy
The discussion of space after a period should include an example
like \verb*|et ``al.''\ | showing where to put the \verb*| | command
when there is no space right after the period.\par}
% John DeTreville, 7 Aug 90
 
\subsubsection*{Section 3.3.3, page 47}
Somewhere, it should be mentioned that each item in an {\tt array}
environment is a separate math formula.
% Walter Taylor, 26 Mar 90
 
 
\subsubsection*{page 50, line 5}
Replace the sentence beginning ``A $+$ or $-$'' with:
\begin{quotation}
A $+$ or $-$ that begins a formula (or certain subformulas) is assumed
to be a unary operator, so typing \verb|$-x$| produces $-x$ (and typing
\verb|$\sum -x_{i}$| produces $\sum -x_{i}$), with no space between the
``$-$'' and the ``$x$''.
\end{quotation}
% Joe Weening, 14 Dec 89
 
\subsubsection*{page 53, line 16}
Change ``math itallic is the default in math mode'' to ``math italic is
almost identical to the default type style of math mode''.
% Matthew Swift, 6 Jun 89.
 
\subsubsection*{page 71, last line}
Change ``comand'' to ``command''.
% William LeFebvre, 2 Jun 88
 
\subsubsection*{page 84}
Add the following paragraph just after line~2.
\begin{quote}
Some commands, such as \verb|\chapter|, change the style of the current
page.  You can change it yourself with the \verb|\thispagestyle|
command.
\end{quote}
Also, add the appropriate index entries for \verb|\thispagestyle| and
\verb|\chapter|.
 
\subsubsection*{page 88, line 29}
Change ``its'' to ``it's''.
% Cynthia Hibbard, 20 Jul 87
 
\subsubsection*{page 91, line 5}
\begin{sloppypar}
It should be noted that in the {\tt book} document style,
\verb|\chapter| uses \verb|\cleardoublepage|.
\end{sloppypar}
 
\subsubsection*{page 111, line $-4$}
Remove the extra {\tt t} from ``{\tt withtout}''.
% Joseph Hurler 25 Aug 87
 
\subsubsection*{page 114, line 31}
Change ``Section 5.6'' to ``Section 5.3''.
% Helmer Aslaksen, 13 Oct 90
 
 
\subsubsection*{page 122, line 17}
Change the sentence that begins ``The most probable cause is'' to
the following:
\begin{quote}
The most probable cause is a list-making environment with no
\verb|\item| command.
\end{quote}
% Oren Patashnik 19 Sep 86
 
 
\subsubsection*{Page 126, line 12ff}
An un\verb|\protect|ed command---even a robust one---in the argument
of one of these commands can cause \TeX\ to run out of this kind of
space.  It's a good idea to \verb|\protect| all commands in
these arguments.
% Flavio Rose, 13 Jun 86
 
 
\subsubsection*{Page 153, top of page}
It should be noted that if the invisible ``word'' occurs at the end
of a paragraph, not attached to a real word, then it could appear
on a line by itself, producing a blank line in the output.
 
\subsubsection*{Page 156}
In the third line of the description of \verb|\footnotemark|, 
remove the phrase ``just like \verb|\footnote|''.
% Stephen Peckham 31 Oct 89
 
\subsubsection*{Page 160, line $-7$}
Replace ``{\tt equations}'' with ``{\tt equation}''.
% Nelson Beebe, 8 Jan 88
 
\subsubsection*{Page 162}
 
In the description of \verb|\pagenumbering|, it should be mentioned
that the command sets the value of the {\tt page} counter to 1.
 
\subsubsection*{Page 161, line $-9$}
Replace ``head information'' by ``head information (but not
page number)''
 
\subsubsection*{Page 162, line 6}
Replace ``{\tt heading}'' by ``{\tt headings}''.
% Mathieu Federspiel 18 Jul 88
 
\subsubsection*{Page 173, line $-2$}
Replace ``{\em cmd\/}'' with ``\verb|\begin{|{\em nam\/}\verb|}|''.
% Michinori Kawasaki, 30 Sep 90.
 
\subsubsection*{Page 177, line 1}
It should be noted that, in a two-column page style, a single-column
figure can come before an earlier double-column figure, and vice-versa.
% Sam Whidden, 5 Jun 86.
 
\subsubsection*{Page 177, line 3}
Replace ``{\em pos\/}'' by ``{\em loc\/}''.
% Steve Kelem, 31 Oct 90
 
\subsubsection*{Page 177, line $-1$}
Replace ``page'' with ``text page''.
% Mike Fischer 20 Mar 87.
 
\subsubsection*{Page 178, lines 1 and 8}
Replace ``page'' with ``text page''.  
% Mike Fischer 20 Mar 87.
 
\subsubsection*{Page 181, line 11}
Replace ``or \verb|\| command'' with ``or \verb|\'| command''.
% Michinori Kawasaki, 30 Sep 90.
 
\subsubsection*{page 183}
The description of \verb|\extracolsep| should mention that no extra
space is put to the left of the first column.
% Scott Simpson 12 May 88.
 
\subsubsection*{page 191}
Add to the description of \verb|\pagebreak| the
information that it has an effect only when used in paragraph mode,
not inside a parbox.
% Andy Hisgen 27 Apr 89
 
\subsubsection*{page 191, line $-2$}
Add a ``t'' to ``explicily''.
% Nelson Beebe 9 Mar 89.
 
\subsubsection*{page 213}
Change the page reference for ``\verb|\bibitem|, moving argument of''
from 153 to 152.
% Oren Patashnik 19 Sep 86
 
\subsubsection*{page 211}
Change index entry for \verb|\AE| to page 40.
% Jerry Leichter 28 Nov 88
 
\subsubsection*{page 213}
Change the first page reference for ``{\tt bbl} (bibliography) file''
from 74 to 75.
% Martin Abadi 14 Mar 88
 
Add page 95 to the entry for ``backspace''.
% Walter Taylor 26 Mar 90
 
\subsubsection*{page 217}
Add an index entry ``diacritical marks, see accents''.
% Suggested by Hank Lewis 22 May 88
 
Move the index entry for \verb|\def| to its proper place on the
preceding page.
% Walter Taylor 26 Mar 90
 
\subsubsection*{page 219, index entry for ``figures''}
Remove the ``and'' from the subentry for ``output by
\verb|\end{document}|''.
% David Wise 5 Aug 86
 
\subsubsection*{page 219, column 1, line 3}
Replace ``anomolous'' by ``anomalous''.
% Martin Abadi 20 Jul 88
 
\subsubsection*{page 223, left column, line 2}
Remove the page number ``79'' from the first subentry
under \verb|\index|.
 
\subsubsection*{page 228}
Change the page reference for ``\verb|\multicolumn|''
from 194  to 184.
% 19 Sep 86
 
\subsubsection*{page 228}
Remove the extra ``,'' from the index entries for \verb|\o|
and \verb|\O|.
% Nelson Beebe 14 Mar 88
 
 
\subsubsection*{page 234}
Add ``48--49'' to the index entry for \verb|\right|.
% Walter Taylor 26 Mar 90

\end{document}
 
%&c&@i*\hbox{* 2@s\|\ @i|}|#&
%&i&\index{#}&
%&m&\mbox#&
%&t&{\tt #}&
%&v&\hbox{\verb|#|}&
%&b&\verb|#|&
%&h&\hbox#&
 
